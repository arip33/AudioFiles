
\section{Server Overview}

In order to negotiate the connection setup between two or more clients, we decided to
design and implement a server for clients to connect to. The server 
facilitates the clients' connections statuses, similar to an instant messaging server. It
provides a method for clients to sign on or off, a way to check friends' online
statuses, and a method to request a friend's ip address and port.

\subsection{Server Design Decisions}

We had to make a number of design decisions for the server, which are described here.

We'll start with the choice of language. Because the server is a separate entity from
the chat client, we had some extra freedom for language and design choice. After all,
the only real requirement for the server was that it be able to connect and communicate
with each client. After some consideration, we decided to implement the server in
Python for two main benefits. First, Python has a number of modules, including a 
socket-based asynchronous server, available to aid with development. Second, Python's
general syntax and ease of use makes it a great tool for creating mock-ups or
relatively small, short-term projects. Of course, it works well for larger projects too.
Python made coding much cleaner and more maintainable for the short term goals of our
project.

We wanted the server to perform some of the same capabilities that a watered-down chat
server (such as one used for Skype) might have or use. To that end, we needed it to
have login functionality and tracking. We determined that the server
had to perform the following tasks at minimum:

\begin{enumerate}
  \item Provide a way for a user to register that he or she is online.
  \item Provide a way for users to talk to each other.
  \item Maintain a list of users each client is permitted to contact.
  \item Provide information updates to each user about their friends' online statuses.
\end{enumerate}

The first task is essential because it represents
our intention to add a server element: to keep track of users.
By giving users a way to register that they are connected (and conversely,
disconnected as they become so), the server could keep track of
users as they log in or log out, allowing it to perform additional functionality. This 
is required for task 4 and the basis of our notification system on the client side. 

The second task is important because, in a typical setting, users aren't going to
know the IP address or port number of the machines their friends are on.
The client needs a way of obtaining this information from the server, and thus being
able to chat with others. It is important that this information is maintained by the
server in order to not limit which machines our users can chat on or where they can 
chat from; therefore, a client must request for this information every time he or she
wants to start a chatting session with one of his or her friends.

Strictly speaking, only points 1 and 2 above are actually needed for a minimalistic
server to function. The server keeps track of who is online, and tells clients where
they can contact other clients who are online. However, we wanted our server to
emulate a couple more features seen in commercial media streaming applications. This 
leads to our third task: maintaining a friends list for each client.
The server would keep a list of who is friends with whom, thereby limiting clients
from being able to randomly call or spam other user online.

The final piece of core functionality is closely related to task 3. In order
to cache friend information on the client side (so the client won't repeatedly make requests
for offline users), the server should be capable of periodically sending updates to each
connected client in order to notify them of which of their friends they can call at 
any given time.

Our final core design choice relates to storage of user information. As it is rather 
standard in industry, the server should be capable of storing information in a database 
file that is easily accessible without any parsing or configurations. In this database, the 
server would keep track of user and friend list information, as well as any other necessary
information needed, i.e. ip addresses and port numbers. Although it isn't very much 
information to keep track of, it still gives a relatively simple solution to keeping track 
of users without.
 
\subsection{Server Implementation}
Now that we've covered the key design decisions of the server, let's describe how the server
was implemented.

Python has a number of different modules available for applications to communicate
through using network sockets. Our first consideration was the SocketServer module. This 
library
provides an interface which can block receive requests, and then execute a request
handler via a callback. Unfortunately, this module has one major flaw: it can only handle
one user request at a time. Because the server needed to handle multiple connected users
asynchronously, we settled on using the asyncore module. This module has essentially
the same functionality as SocketServer, but with asynchronous I/O. In other words, each
new incoming request executes a request handler in its own seperate thread, allowing 
multiple clients to connect and remain logged in at the same time.

In order to allow clients to log on, request information, and receive updates, it made
sense for the server to use a reliable connection-oriented protocol. Thus, TCP was the 
natural
choice. When a client connects to the server, a new thread is spawned to maintain
the connection to the client. If the client makes any requests, the response packets will 
be sent
directly to this child thread instead of the main server loop. Each child thread has its own
instance of a request handler that handles the logic for reading, interpreting, and replying
to each client request. If a client closes its connection, the thread closes as well and is
removed from the loop. 

To log in, a client establishes a connection to the server (in our tests
the client would accept the IP address of the server as an input parameter) and sends it
a packet with two 20 byte strings, a username and a password, padded as necessary with
null bytes. The server then accepts the packet, verifies the username and password, and
replies back to the client with either a success or failure packet depending on the
validity of the credentials. The password is sent over the network as plaintext (normally
it would be encapsulated in an SSL connection) and the server, upon reception, hashes
the password with salt using an SHA-512 function and compares the result to the truth value 
in the database. If the hashes match, the user is granted login priviledges. This
security isn't hack-proof, but it at least models a standard login procedure.  Additional,
security concerns are left for future work.

Once a client successfully logs in, the server replies back with a friend list packet.
This packet contains a list of usernames and ids of all the users a client is currently
friends with. This packet does not contain any online statuses, and in the event of
an overflow (more friends than can fit in the packet) the server would simply truncate
the list. We did not run into any issues with this in testing because we only had 2
users, but in a real-world environment this certainly would be more of a concern and 
therefore is left for future work.

Every few seconds (our implementation configured it to 4), the server sends out an
update to each client currently maintaining a connection to it. To accomplish this,
the server initializes a timer to provide an interrupt which executes a special method
once per interval. This method, when run, determines who has changed status in the past
interval (whether they went online or offline), and sends each of the clients' friends a 
status update. To reduce the overhead of using the same
connection, this status update would be sent via a separate connection that each client
establishes when it first connects. When the method finishes executing, it uses a callback
to set up a new timer interrupt. 

One other capability the server provides to each client is a friend address request. Due to
the nature of our project and the desire to not want to endlessly enter IP addresses and
port information
when performing tests, we implemented the ability for a client to request a friend's IP 
address and port number
information from the server. A client simply sends a packet containing the friend's ID and 
the server, upon verifying the friend is indeed a friend and is online, replies with the
IP address and the port number of that friend's computer. This information is then used
by the client to establish a client-to-client connection that is independent of the 
server.

For sending and receiving binary data over the network, the server uses the Python struct
module, which provides a convenient way to pack and unpack data with specific size
limits. When sending data, the server calculates the values it needs, packs them as
appropriate, and concatenates the values together (the struct module uses strings
for its binary representation, in which an encoded character represents the specific values
we want to send or receive). When receiving data, the server decrypts this encoded character
to identify the type of packet received.

To keep track of users and friends, the server uses a SQLite database consisting of
two tables. The first table, called ``Users,'' has one row per user, with columns
giving the username, user ID, and password hash. The second table, called ``Friends,''
contains a row per user which includes each of the users friends. The design of this is such
that ``A is friends with B'' and ``B is friends with A'' needs to occur in the table
for A and B to properly communicate. In other words, for every set of friends, two rows
in the ``Friends'' table are needed. This is not exactly the most efficient (or even the
most reliable) way to keep track of the information, but it served its purpose well enough
for the scope of this project. When the server wants to obtain information from the
database, it uses the sqlite3 Python module to do so. 
